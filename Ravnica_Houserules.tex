\documentclass[12pt]{article}

\usepackage{amssymb}
\usepackage{amsmath}
\usepackage[utf8]{inputenc}
\usepackage[ngerman]{babel}
\usepackage{lineno}
\usepackage{listings}
\usepackage[T1]{fontenc}
\usepackage[utf8]{inputenc}
\usepackage{lmodern}
\usepackage{eurosym}
\usepackage{listings}
\usepackage{microtype}
\usepackage{units}
\usepackage{color}
\usepackage{xcolor}
\usepackage{graphicx}
\usepackage{subfigure}
\usepackage{import}
\usepackage{url}
\usepackage{amsthm}
\theoremstyle{plain}

\lstset
{ %
  language=R,                     % the language of the code
  basicstyle=\footnotesize\ttfamily,       % the size of the fonts that are used for the code
  numbers=left,                   % where to put the line-numbers
  numberstyle=\tiny\color{gray},  % the style that is used for the line-numbers
  stepnumber=1,                   % the step between two line-numbers. If it's 1, each line
                                  % will be numbered
  numbersep=5pt,                  % how far the line-numbers are from the code
  backgroundcolor=\color{white},  % choose the background color. You must add \usepackage{color}
  showspaces=false,               % show spaces adding particular underscores
  showstringspaces=false,         % underline spaces within strings
  showtabs=false,                 % show tabs within strings adding particular underscores
  frame=single,                   % adds a frame around the code
  rulecolor=\color{black},        % if not set, the frame-color may be changed on line-breaks within not-black text (e.g. commens (green here))
  tabsize=2,                      % sets default tabsize to 2 spaces
  captionpos=b,                   % sets the caption-position to bottom
  breaklines=true,                % sets automatic line breaking
  breakatwhitespace=false,        % sets if automatic breaks should only happen at whitespace
  title=\lstname,                 % show the filename of files included with \lstinputlisting;
                                  % also try caption instead of title
%  keywordstyle=\color{blue},      % keyword style
%  commentstyle=\color{green},   % comment style
%  stringstyle=\color{blue},       % string literal style
  %escapeinside={\%*}{*)},         % if you want to add a comment within your code
  escapeinside={(*@}{@*)},         
  morekeywords={*,...}            % if you want to add more keywords to the set
} 

\title{\vspace{-2cm}Blatt 01}
\author{Kolja Hopfmann, Julian Göttmann, Jonas Sander}
\date{\today}

\begin{document}
\pagenumbering{arabic}
\maketitle
\centerline{\rule{1.2\linewidth}{.2pt}}
%\tableofcontents
\shorthandoff{"}
\section{Rassen}
\subsection{Faerie}
The fae of Lorwyn lead short, flitting lives in pursuit of gossip, diversions, and amusing intrigues. But faeries can also be carelessly cruel, capricious, and vindictive. The faeries travel in small groups of three to six called cliques. It is thought that faeries do not dream, which would explain why they spend so much time harvesting the dreams of others. Faeries can distill these stolen dreams into a sparkling energy that they carry around with them. \\
\url{https://mtg.gamepedia.com/Faerie}
\section{Klassen}
\subsection{Spellblade}
Wie Paladin, nur mit Zugriff auf Wizard Spells? Multiclassing?
\section{Mobs}
\subsection{Azorius}
Archetype: Spellblade/Paladin/Cleric, Fokussiert auf Magieaufhebung und Kontrolle, Polizei
\subsection{Boros}
Archetype: Fighter, Armee
\subsection{Dimir}
Archetype: Rougue/Spellblade, Fokussiert auf Illusion, Schleichen, Geheimdienst
\subsection{Golgari}
Archetype: 
\subsection{Gruul}
Archetype:
\subsection{Orzhov}
Archetype:
\subsection{Izzet}
Archetype: Wizard/Sorcerer, Fokussiert auf Elementarmagie, Artefakte, Uni/Forschungszentrum
\subsection{Rakdos}
Archetype: 
\subsection{Selesnya}
Archetype:
\subsection{Simic}
Archetype: Wizard/Sorcerer/Warrior/Rogue, Merfolk
\end{document}